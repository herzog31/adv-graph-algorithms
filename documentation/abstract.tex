%==================================================
% abstract.tex
% Beispieldatei für tumthesis.cls und thesis.tex
% Michael Ritter, 2012
% Lizenz: 
% This work may be distributed and/or modified under the
% conditions of the LaTeX Project Public License, either version 1.3
% of this license or (at your option) any later version.
% The latest version of this license is in
% http://www.latex-project.org/lppl.txt
% and version 1.3 or later is part of all distributions of LaTeX
% version 2005/12/01 or later.
%==================================================
\cleardoublepage

\selectlanguage{english}
\section*{Abstract}
Here we give a short summary of the project or thesis of length at most a quarter of a page. This could be \eg as follows:

This document is an introduction to the use of the \LaTeX-package \texttt{tumthesis.cls}, with which theses can be written in the TUM style. The basic structure of the example files is explained and some optional components are mentioned briefly. There are also some tips for \LaTeX beginners (and also for more advanced users who want to learn some more) as well as suggested reading for individual study.


\selectlanguage{ngerman}
\section*{Zusammenfassung}
Hier schreibt man eine kurze Zusammenfassung der Arbeit im Umfang von maximal einer Viertelseite. Das kann \eg so aussehen:

Die Arbeit führt in die Verwendung des \LaTeX-Pakets \texttt{tumthesis.cls} ein, mit dem Abschlussarbeiten im TUM-Stil gesetzt werden können. Die grundlegende Gliederung der Beispieldateien wird erklärt und auf optionale Bestandteile wird kurz eingangen. Außerdem enthält der Text ein paar Tipps für \LaTeX-Anfänger (und auch für Fortgeschrittene, die noch etwas dazulernen wollen) sowie Literaturhinweise zum Selbststudium.

\selectlanguage{ngerman}

%%% Local Variables: 
%%% mode: latex
%%% TeX-master: "thesis"
%%% End: 