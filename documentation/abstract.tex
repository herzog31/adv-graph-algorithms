%==================================================
% abstract.tex
% Beispieldatei für tumthesis.cls und thesis.tex
% Michael Ritter, 2012
% Lizenz: 
% This work may be distributed and/or modified under the
% conditions of the LaTeX Project Public License, either version 1.3
% of this license or (at your option) any later version.
% The latest version of this license is in
% http://www.latex-project.org/lppl.txt
% and version 1.3 or later is part of all distributions of LaTeX
% version 2005/12/01 or later.
%==================================================
\cleardoublepage

%\selectlanguage{english}
%\section*{Abstract}
%Here we give a short summary of the project or thesis of length at most a quarter of a page.


\selectlanguage{ngerman}
\section*{Zusammenfassung}
Das interdisziplinäre Projekt beschäftigt sich mit der anschaulichen Darstellung von weiterführenden Graphalgorithmen. Betrachtet werden realitätsnahe Probleme, wie das All-Pairs Shortest Path Problem, Matchingprobleme in bipartiten Graphen, das Eulertour Problem und das Chinese Postman Problem. Alle Problemstellungen haben gemeinsam, dass sich die zur Lösung verwendeten Algorithmen anschaulich darstellen lassen.

Das Ziel des Projekts ist, die erwähnten Problemstellungen mit einfachen Worten zu vermitteln, sowie die verwendeten Lösungsverfahren interaktiv zu veranschaulichen. Die Darstellung erfolgt in Form mehrerer Web-Applikationen, welche aus Kontinuitätsgründen auf ein gemeinsames Framework aufbauen, welches bereits bei früheren Projekten zum Einsatz kam. In den Applikationen wird der Benutzer zunächst an die jeweilige Problemstellung herangeführt. Anschließend hat er die Möglichkeit die Algorithmen auf selbst erstellen Graphen schrittweise ausführen zu lassen. Die speziellen Eigenheiten der Algorithmen werden weiterhin in gesonderten Forschungsaufgaben behandelt. \hfill$\spadesuit$