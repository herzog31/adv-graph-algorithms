%% conclusion.tex
%%

%% ==================
\chapter{Zum Weiterlesen}
\label{ch:reader}
%% ==================

In diesem Kapitel sammeln wir ein paar Literaturhinweise in Sachen \LaTeX, die
Anfängern und Fortgeschrittenen Anwendern den Einstieg erleichtern und
vielleicht einige nützliche Hinweise geben können.
\begin{description}
\item[lshort:] \enquote{The Not So Short Introduction to \LaTeX}
  (vgl. \cite{l2short}) ist eine aktuelle Einführung, die sich mit moderatem
  Zeitaufwand zum Einstieg durcharbeiten lässt (die Autoren geben für die zum
  Zeitpunkt der Erstellung dieses Dokuments aktuelle Version 5.01 eine Zeit von
  157 Minuten an). Die jeweils aktuelle Version findet man auf
  \url{http://tobi.oetiker.ch/lshort/lshort.pdf}.
\item[\LaTeX{} and Friends:] Das Buch \cite{vanDongen2012} ist eine
  empfehlenswerte und aktuelle Einführung in  \LaTeX, die viele aktuelle Pakete
  behandelt. Für Einsteiger wie auch für Fortgeschrittene einen Blick wert.
\item[l2tabu:] In \cite{l2tabu} sind einige Hinweise auf veraltete Pakete und
  \LaTeX-Befehle enthalten, die besonders für Fortgeschrittene \LaTeX-Nutzer
  lesenswert sind. Hier erfährt man, warum man bestimmte Kommandos besser nicht
  einsetzen sollte und was bessere Alternativen sind. Übrigens: Die Klasse
  \texttt{tumthesis.cls} bindet automatisch das Paket \texttt{nag} ein, das bei
  vielen in l2tabu gelisteten Fehler direkt Alarm schlägt.
\end{description}

%%% Local Variables: 
%%% mode: latex
%%% TeX-master: "thesis"
%%% End: 
