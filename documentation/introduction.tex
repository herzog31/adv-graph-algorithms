\chapter{Einleitung}
Einleitung
Motivation
Referenz auf führere IDPs (Quellen!)
didaktisches Konzept, kurz
Aufbau der Dokumentation, inkl. Benennung wer hat was gemacht (Benotung)

% Keine Beschreibung der Algorithmen an sich!

\chapter{Aufbau der Webanwendungen}
funktionale Beschreibung der Apps
jedes Tab

\section{Algorithmus von Floyd-Warshall}
Besonderheiten Visualisierung
Forschungsfragen

\section{Algorithmus von Hierholzer (Mark)}
Besonderheiten Visualisierung
Forschungsfragen

\section{Algorithmus von Hopcroft und Karp}
Besonderheiten Visualisierung
Forschungsfragen

\section{Ungarische Methode}
Besonderheiten Visualisierung
Forschungsfragen

\section{Chinese Postman Problem}
Besonderheiten Visualisierung
Forschungsfragen

\chapter{Implementierung}

\section{Installation (Mark)}
Zur Installation sind keine speziellen Anforderungen zu erfüllen. Die Webapplikationen wurden vollständig mittels HTML5, JavaScript und CSS implementiert, sodass keine zusätzliche serverseitige Software benötigt wird. Zum Aufrufen der Anwendungen wird ein moderner Webbrowser benötigt.

Die Bereitstellung erfolgt über den Online Dienst GitHub, der das verteile Versionskontrollsystem Git benutzt. 
Alle Anwendungen liegen in einem gemeinsamen Repository, welches unter \url{https://github.com/herzog31/adv-graph-algorithms} erreichbar ist. Zur Installation kann man entweder auf der Repository Seite die letzte freigegebene Version (Release) als Zip oder Tar Archiv herunterladen oder die aktuellste Version des Repositories mittels folgendem Befehl in das aktuelle Verzeichnis kopieren.

\begin{figure}[htb]
	\begin{lstlisting}[language=Bash]
    git clone -b master https://github.com/herzog31/adv-graph-algorithms.git
  	\end{lstlisting}
  	\caption[Repository Kopieren]{Befehl zum Kopieren des GitHub Repositories}\label{fig:sample-listing}
\end{figure}

Um die Anwendungen als Webanwendungen online bereitzustellen wird ein Webserver benötigt. Hier emfiehlt sich die Installation des Apache oder nginx  HTTP Servers.

Die lokale Ausführung ist ohne einen Webserver möglich. Verschiedene Webbrowser besitzen allerdings eine Sicherheitsrichtlinie, die das Öffnen von lokalen Dateien über JavaScript verbieten. Diese Sicherheitsrichtline kann jedoch durch spezielle Einstellungen umgangen werden. Für Google Chrome ist dies über den Start Parameter \texttt{--allow-file-access-from-files} möglich.

\section{MathJAX (Mark)}
Motivation
Beispiel

\section{Bipartite Graphen}
Motivation
Beispiel

\section{Multigraphen}
Motivation
Beispiel

\section{Zufällig generierte Fragen (Mark)}
Motivation
Beispiel

\section{Gemeinsam genutzte Dateien (Mark)}
Motivation
Beispiel

\chapter{Zusammenfassung}
Zusammenfassung der wesentlichen Punkte