%% ==============================
\chapter{Einleitung}
\label{ch:introduction}
%% ==============================
Dieses Kapitel gibt ein kurzes Beispiel für die Verwendung der
\texttt{tumthesis}-Klasse. 

\section{Dateien}
\label{sec:intro:dateien}

\Cref{tab:intro:dateien} zeigt alle zu diesem Beispiel
gehörenden Dateien mit einer kurzen Erklärung.

\begin{table}[htb]
  \centering
  \begin{tabular}{lp{10cm}} 
    \toprule
    \textbf{Dateiname} & \textbf{Erklärung} \\ \midrule
    \texttt{tumthesis.cls} & Klassendatei, stellt die grundlegenden Kommandos
    bereit und bindet wichtige Pakete ein\\
    \texttt{tumcolors.sty} & \LaTeX-Paket, in dem die offiziellen Farben der TUM
    definiert werden, wird von tumthesis.cls benutzt\\
    \texttt{thesis.tex} & Hauptdatei dieses Beispiels, Ausgangspunkt für die
    eigene Arbeit. Alle anderen \texttt{.tex}-Dateien werden von dieser Datei
    eingebunden.\\
    \texttt{thesis.pdf} & PDF-Version von thesis.tex\\
		\texttt{preamble.tex} & Präamble, in welcher vom Nutzer Pakete geladen werden können\\
    \texttt{abstract.tex} & Text für die Abstracts in Deutsch und Englisch\\
    \texttt{introduction.tex} & Text für dieses Kapitel\\
    \texttt{conclusion.tex} & Text für das folgende Kapitel\\
    \texttt{appendix.tex} & Text für den Anhang\\
    \texttt{thesis.bib} & Bib\TeX-Datei für die Literaturhinweise\\
    \texttt{TUMlMblau} & Logo der Mathematik-Fakultät für die Titelseite\\
    \texttt{TUMloblauSchriftzugL} & TUM-Logo für die Titelseite\\
		\texttt{TopMath-Bildmotiv.jpg} & TopMath Logo für die Titelseite\\
    \bottomrule
  \end{tabular} 
  \caption{zum Beispiel gehörige Dateien}
  \label{tab:intro:dateien}
\end{table}

Um aus den Quelldateien das fertige PDF zu erstellen, sind folgende Befehle
aufzurufen:
\begin{verbatim}
pdflatex thesis
biber thesis
makeindex -s myindex.ist
pdflatex thesis
pdflatex thesis
\end{verbatim}

Der erste Durchlauf von \texttt{pdflatex} erzeugt einige Hilfsdateien und eine
(fast schon fertige) PDF-Ausgabe -- lediglich einige Grafiken können noch an der
falschen Stelle sitzen und die Verweise und Literaturangaben funktionieren noch
nicht. Mit dem \texttt{biber}-Aufruf arbeitet das System dann die Datei
\texttt{thesis.bib} durch und erzeugt daraus das Literaturverzeichnis (siehe \cref{sec:intro:biblatex}). Anschließend wird mit \texttt{makeindex} noch der Index erstellt (siehe \cref{sec:intro:index}). Die
beiden folgenden Aufrufe von \texttt{pdflatex} dienen dazu, die Verweise korrekt
zu setzen und die Platzierung der Grafiken auszurichten. 

Fortgeschrittene können den Prozess übrigens auch automatisieren, indem einfach
\begin{verbatim}
latexmk --pdf thesis
\end{verbatim}
aufgerufen wird. Das Tool \texttt{latexmk} sorgt dann automatisch für die
korrekte Anzahl von Aufrufen und für den Bib\TeX-Durchlauf.

\section{Konfiguration und Optionen}
\label{sec:intro:optionen}
Eigene Pakete des Nutzers sowie Änderung von vorgenommenen Einstellungen können in der Datei \texttt{preamble.tex} geladen werden. Der dort enthaltene Code wird zum Ende der Klasse (unmittelbar vor dem Laden der letzten Pakete \texttt{hyperref}und \texttt{cleveref}) eingelesen und ausgeführt.

Die \texttt{tumthesis}-Klasse akzeptiert einige Optionen, die das Aussehen der Titelseite und anderes Verhalten ändern:
\begin{itemize}
	\item \texttt{topmath}: Diese Option platziert das TopMath-Bildmotiv auf der Titelseite:
		\begin{verbatim}
		\documentclass[topmath]{tumthesis}
		\end{verbatim}
\end{itemize}
Alternativ kann man auch ein eigenes Motiv auf der Titelseite platzieren:
\begin{itemize}
	\item \texttt{titlepicture}: Name der einzubindenen Datei
	\item \texttt{titlepictureX}: Horizontaler Abstand (mit Einheit) zwischen rechter unteren Ecke der Seite und recher unteren Ecke des Bildes
	\item \texttt{titlepictureY}: Vertikaler Abstand (mit Einheit) zwischen rechter unteren Ecke der Seite und recher unteren Ecke des Bildes
		\begin{verbatim}
		\documentclass[titlepicture=MA_CMYK.pdf,titlepictureX=25mm,
				titlepictureY=40mm]{tumthesis}
		\end{verbatim}
        würde nochmals das Mathelogo rechts unten auf der Seite einfügen.
\end{itemize}
Weiterhin kann auch das Verhalten des Inhalts von Theoremen angepasst werden:
\begin{itemize}
	\item \texttt{theoremtitle}: Soll der Inhalt eines Theorems neben dem Titel begonnen werden (nobreak) oder in der nächsten Zeile (break, Standardoption)
		\begin{verbatim}
		\documentclass[theoremtitle=nobreak]{tumthesis}
		\end{verbatim}
\end{itemize}
Eine weitere Option erlaubt die Konfiguration des BibLaTeX-Backends:
\begin{itemize}
	\item \texttt{biblatexBackend}: Die Defaulteinstellung ist Biber (Parameter biber), alternativ sind alle Optionen des Parameters \enquote{backend} des BibLaTeX-Packets möglich:
		\begin{verbatim}
		\documentclass[biblatexBackend=bibtex]{tumthesis}
		\end{verbatim}
\end{itemize}

\section{Grundeinstellungen}
\label{sec:intro:grundeinstellungen}

Ganz zu Beginn werden in der Datei \texttt{thesis.tex} ein paar wichtige
Einstellungen vorgenommen. Der Code sieht wie folgt aus:

\begin{lstlisting}[language={[LaTeX]TeX}]
% -------------------------------
% PDF-Information
\hypersetup{
 pdfauthor={Wolfgang Ferdinand Riedl, Michael Ritter},
 pdftitle={Die tumthesis-Klasse},
 pdfsubject={Anleitung für Abschlussarbeiten},
 pdfkeywords={Masterarbeit, Bachelorarbeit}
 colorlinks=true, %farbige Links (für die PDF-Version)
% colorlinks=false, % keine farbigen Links (für die Druckversion)
}

% -------------------------------

% Basisdaten 

\author{Wolfgang F. Riedl, Michael Ritter}
\title{Die \texttt{tumthesis}-Klasse}
\subtitle{Eine Anleitung für Abschlussarbeiten}
\faculty{Fakultät für Mathematik}
\institute{Lehrstuhl für Angewandte Geometrie und Diskrete Mathematik}
%\subject{master}
%\subject{bachelor}
%\subject{diploma}
%\subject{project}
%\subject{seminar}
%\subject{idp}
\subject{Kleiner Überblick}
\professor{Prof. Dr. Peter Gritzmann} %Themensteller
\advisor{Dr. René Brandenberg} %Betreuer
\date{26.12.2012} %Abgabedatum
\place{München} %Ort für die Unterschrift
\end{lstlisting}

Die Angaben im \verb|hypersetup|-Befehl erscheinen zwar nicht im Dokument
selbst, sie werden aber als Metadaten in die PDF-Datei eingebettet und lassen
sich im Acrobat Reader (und vielen anderen PDF-Betrachtern) anzeigen. Bei den
restlichen Befehlen dürfte das meiste selbsterklärend sein. Beim
\verb|subject{}| kann man entweder einen beliebigen Text angeben (wie das in
diesem Beispiel mit \enquote{Kleiner Überblick} gemacht wird) oder man verwendet
eines der vordefinierten Schlüsselwörter, die automatisch die Ausgabe
\enquote{Masterarbeit}, \enquote{Bachelorarbeit} bzw. eine passende andere
Ausgabe erzeugen.  Dazu entfernt man einfach das Kommentarzeichen in der
betreffenden Zeile und kommentiert dafür den jetzt aktiven \verb|subject|-Befehl
aus. Es ist natürlich darauf zu achten, dass man die Bezeichnungen der gewählten
Sprache anpasst, vgl. dazu \cref{sec:intro:sprachauswahl}. (Man kann den
\verb|\subject{}|-Befehl übrigens auch ganz weglassen, dann wird eben keine
Bezeichnung auf der Titelseite erzeugt und nur der Autor genannt.)

\section{Sprachauswahl und Zeichensatz}
\label{sec:intro:sprachauswahl}
Die Klasse unterstützt als Sprachen Deutsch und Englisch. Zu Beginn von
\texttt{thesis.tex} gibt es den Befehl
\begin{lstlisting}[language={[LaTeX]TeX}]
  \selectlanguage{ngerman}
\end{lstlisting}
mit dem die Grundsprache eingestellt wird. Diese kann man mit genau diesem
Befehl bzw. \verb|\selectlanguage{english}| übrigens jederzeit im Dokument
ändern. Dabei passen sich ein paar Einstellungen automatisch an, \eg liefern die
Befehle \verb|\eg| und \verb|\ie| jeweils den passenden Text (die sollte man
übrigens verwenden, weil sie gleich für den typographisch korrekten Abstand
sorgen) und die Überschriften ändern sich, aber auch etwas subtilere Dinge wie
die verwendeten Trennmuster für die automatische Silbentrennung. Ein Beispiel
für so eine Umschaltung findet man im Abstract.

Für eigene Dateien ist es wichtig, das korrekte \enquote{Encoding} zu
wählen. Vorgabe ist hier Unicode (UTF-8). Das ermöglicht es, Umlaute und andere
Sonderzeichen direkt einzugeben, erfordert aber unter Umständen eine richtige
Einstellung im Editor. Besonders unter Windows-Systemen sind manche Editoren
standardmäßig auf Latin-1 statt auf Unicode eingestellt -- so etwas kann zu sehr
merkwürdigen Fehlermeldungen führen!

\section{Druck}
\label{sec:intro:bindung}
Beim Drucken ist unbedingt darauf zu achten, dass die Arbeit doppelseitig
gedruckt wird. Die Seitenränder und die Kopf- und Fußzeilen sind darauf
ausgelegt, dass die Arbeit doppelseitig gedruckt und dann gebunden wird.

Um die Bindekorrektur anzupassen (und so zum Beispiel in der Mitte mehr Platz für die Bindung zu lassen) muss einfach die Zeile 
\begin{lstlisting}[language={[LaTeX]TeX}]
  BCOR =5 mm % Binding correction , ensures sufficient space for binding
\end{lstlisting}
in der Datei \verb|tumthesis.cls| angepasst werden.

\section{Titelseite}
Standardmäßig wird eine dem TUM-Styleguide weitestmöglich folgende Titelseite dargestellt. Durch das Ersetzen der Zeile
\begin{verbatim}
\maketitlepage%
\end{verbatim}
in der Datei \texttt{thesis.tex} durch
\begin{verbatim}
\maketitlepageDissertation%
\end{verbatim}
wird diese durch eine an eine Dissertation angepasst Titelseite ersetzt.

\section{Wichtige Hinweise}
\label{sec:intro:hinweise}
\subsection{Mathe-Umgebungen}
Da diese Klasse das Packet \texttt{ntheorem} lädt, werden durch Matheumgebungen der Art 
\verb|\[ ... \]| Fehlermeldungen erzeugt. Solche Umgebungen sollten stattdessen durch
\verb|\begin{equation*} ... \end{equation*}| ersetzt werden.

\subsection{BibLaTeX}
\label{sec:intro:biblatex}
Das Packet nutzt als Standard das Biber Backend, welches UTF-8 kodierte Bibliographiedateien 
lesen kann. Die Umstellung auf ein anderes Backend erfolgt über den Parameter \texttt{biblatexBackend} (siehe Parameter).

Für die meisten \LaTeX-Editoren findet man im Web Anleitungen für die Einrichtung mit biber.
\section{Einige Pakete}
\label{sec:intro:pakete}
Die Klasse \texttt{tumthesis.cls} bindet bereits eine ganze Reihe nützlicher
Pakete ein, die wir hier kurz auflisten wollen.

\subsection{Index}
\label{sec:intro:index}
Die tumthesis Klasse lädt das Paket imakeidx, welches die schnelle und einfach Erstellung eines \emph{Index}\index{Index}\index{index!imakeidx} erlaubt. Um ein Wort zum Index hinzuzufügen muss einfach der Befehl \verb|\index{Schlüsselwort}| angehängt werden. Das Wort \enquote{Index} wird zum Beispiel wie folgt zum Index dieses Dokuments hinzugefügt:
\begin{lstlisting}[language={[LaTeX]TeX}]
  ... einfach Erstellung eines \emph{Index}\index{Index} erlaubt. Um ...
\end{lstlisting}
Symbole können ebenso einfach hinzugefügt werden: Das Symbol $\zeta$\index{$\zeta$} wird mit folgender Zeile zum Index hinzugefügt:
\begin{lstlisting}[language={[LaTeX]TeX}]
  ... Das Symbol $\zeta$\index{$\zeta$} wird mit ...
\end{lstlisting}
Um die Position eines Symbols (oder anderen Elements) im Index zu ändern, kann man ein zusätzliches Schlüsselwort (welches auch eine Formel sein kann) angeben:

Um das Symbol $\pi$\index{pi@$\pi$}\index{$p_i$@$\pi$} so in den Index aufzunehmen, dass es an der Stelle des Wortes \enquote{pi} und zusätzlich noch an der Position des Symbols \enquote{$p_i$} erscheinen würde, kann man den folgenden Code nutzen:
\begin{lstlisting}[language={[LaTeX]TeX}]
  ... as Symbol $\pi$\index{pi@$\pi$}\index{$p_i$@$\pi$} so in den ...
\end{lstlisting}

Subkategorien können auch sehr einfach über \verb|\index{Schlüsselwort!Subkategorie}| erstellt werden: Die Definition eines metrischen Raums 
\begin{definition}[Metric]
	\index{metric}
	Let $X$ be a set and $d: X \times X \longrightarrow \mathbb{R}$. The function $d$ is a metric on $X$ if the following three properties hold for all $x,y,z \in X$
	\begin{enumerate}
	  \item $d(x,x) \geq 0$ and $d(x,y) = 0 \iff x = y$ (non-negativity)\index{metric!non-negativity}
	  \item $d(x,y) = d(y,x)$ (symmetry)\index{metric!symmetry}
	  \item $d(x,z) \leq d(x,y) + d(y,z)$ (triangle inequality)\index{metric!triangle inequality}.
	\end{enumerate}
\end{definition}
kann beispielsweise im Index mit Hilfe des folgenden Codes referenziert werden:
\begin{lstlisting}[language={[LaTeX]TeX}]
\begin{definition}[Metric]
	\index{metric}
	Let $X$ be a set and $d: X \times X \longrightarrow \mathbb{R}$. The function $d$ is a metric on $X$ if the following three properties hold for all $x,y,z \in X$
	\begin{enumerate}
	  \item $d(x,x) \geq 0$ and $d(x,y) = 0 \iff x = y$ (non-negativity)\index{metric!non-negativity}
	  \item $d(x,y) = d(y,x)$ (symmetry)\index{metric!symmetry}
	  \item $d(x,z) \leq d(x,y) + d(y,z)$ (triangle inequality)\index{metric!triangle inequality}.
	\end{enumerate}
\end{definition}
\end{lstlisting}

Zur Erstellung des Index muss die folgende Zeile
\begin{lstlisting}[language={[LaTeX]TeX}]
	\makeindex[title=Index,options=-s myindex]
\end{lstlisting}
zur Datei \verb|thesis.tex| \emph{vor} dem Befehl \verb|\begin{document}| hinzugefügt werden! Die Option \verb|title=Index| gibt die Überschrift des Index (in diesem Fall \enquote{Index}) an; die zweite Option eine angepasst Style-Datei (funktioniert nicht in allen Umgebungen, andernfalls müssen die unten aufgeführten Kommandozeilenoptionen genutzt werden).

Der Index kann dann über das folgende Kommando zum Dokument hinzugefügt werden:

\begin{lstlisting}[language={[LaTex]TeX}]
	%Add the index to the table of contents
	\addcontentsline{toc}{chapter}{Index}
	%print the index
	\printindex
\end{lstlisting}
Der Index muss mit Hilfe des \verb|makeindex|-Befehl kompiliert werden (was die meisten Editoren automatisch machen). Das Layout des Index kann mit Hilfe einer modifizierten Style-Datei angepasst werden, diese wird über die Option \verb|-s stylefile.ist| geladen (oder unter Umständen über die oben dem \verb|\makeindex|-Befehl übergebene Option). Dieses Dokument wurde mit Hilfe des in der beiliegenden Datei d\verb|myindex.ist| definierten Styles erstellt.

\subsection{scrbook}
\label{sec:intro:scrbook}
Die Klasse \texttt{tumthesis.cls} baut komplett auf \texttt{scrbook.cls}
auf. Das bedeutet insbesondere, dass auch alle Optionen und Befehle von
\texttt{scrbook} zur Verfügung stehen. Genaueres dazu findet man in der
Dokumentation \cite{KohmMorawski2012} oder in gedruckten Ausgabe
\textcite{KohmMorawski2012b}.

\subsection{csquotes}
\label{sec:intro:csquotes}
Dieses Paket liefert unter anderem den Befehl \verb|\enquote{}|, mit dem sich
automatisch korrekte Anführungszeichen setzen lassen. Dabei richtet sich das
Paket nach der gerade aktiven Sprache: In deutschen Texten erscheinen
\enquote{deutsche Anführungszeichen}, \selectlanguage{english} while English
texts use \enquote{corresponding quotation marks}. \selectlanguage{ngerman}

\subsection{cleveref}
\label{sec:intro:cleveref}
Verweise unter \LaTeX{} setzt man normalerweise mit \verb|\ref{}|. Dieses Paket
definiert die neuen Befehle \verb|\cref| und \verb|\Cref|, die dafür sorgen,
dass neben der richtigen Nummer automatisch auch ein beschreibender Text gesetzt
wird (und zwar in der gerade eingestellten Sprache). Die zweite Version sorgt
dabei für Großschreibung und sollte daher am Satzanfang verwendet werden (auch
wenn das im Deutschen meistens keinen Unterschied macht, weil die Bezeichnungen
häufig Substantive sind, die ohnehin großgeschrieben werden). Ein Beispiel kann
man weiter oben und auch hier sehen: Die Verweise auf \cref{tab:intro:dateien}
werden mit \texttt{cleveref} erzeugt, der Text \enquote{Tabelle} wird dabei
automatisch eingefügt.

\subsection{ntheorem}
\label{sec:intro:ntheorem}
Mit diesem Paket werden eine Reihe Standard-Umgebungen für Definitionen, Sätze,
Beweise etc. bereitgestellt. Die Bezeichnungen sind übrigens sprachabhängig. Ein
Beispiel:
\begin{definition}
  Jedes Element eines Vektorraums bezeichnen wir als \emph{Vektor}.
\end{definition}

\begin{satz}[Hauptsatz der kanonischen Vektorraumbezeichnungen]
  \label{satz:vektorhauptsatz}
  Zu jedem Vektor $v$ gibt es einen Vektorraum $V$ mit $v \in V$.
\end{satz}
\begin{beweis}
  Der triviale Beweis wird dem Leser zu Übung überlassen. Es ist wirklich nicht
  schwer, probieren Sie es einfach. 
\end{beweis}

\selectlanguage{english}
Let us now demonstrate an English version of the above proof:
\begin{beweis}
  The proof of \cref{satz:vektorhauptsatz} is most trivial and only complete
  idiots would not be able to do it themselves. If you even bothered reading
  this proof you might want to think about studying some other subject.
\end{beweis}
\selectlanguage{ngerman}

Wir formulieren noch einen Satz, um nebenbei ein weiteres Feature von
\texttt{cref} vorzuführen, mit dem man nämlich auch mehrere Verweise
zusammenpacken kann, hierzu sei auf \cref{satz:vektorhauptsatz,satz:latex}
verwiesen.
\begin{satz}
\label{satz:latex}
  \LaTeX{} ist toll!
\end{satz}

Die Umgebungen lassen sich natürlich auch ergänzen und dem eigenen Geschmack
anpassen. Dazu kann entweder die Beispiele in \texttt{tumthesis.cls} anschauen
oder man zieht die Dokumentation \cite{ntheorem} zum \texttt{ntheorem}-Paket zu
Rate. In \cref{tab:ntheorem} sind alle vordefinierten Umgebungen aufgelistet.

\begin{table}[hbt]
  \centering
  \begin{tabular}{lll}
    \toprule%
    \textbf{Umgebung} & \multicolumn{2}{c}{\textbf{Text}} \\
    & \textbf{Deutsch} & \textbf{Englisch}\\ \midrule
    \texttt{definition} & Definition & Definition \\
    \texttt{theorem} & Satz & Theorem \\
    \texttt{satz} & Satz & Theorem \\
    \texttt{lemma} & Lemma & Lemma\\
    \texttt{proposition} & Proposition & Proposition \\
    \texttt{corollary} & Korollar & Corollary\\
    \texttt{korollar} & Korollar & Corollary\\
    \texttt{remark} & Bemerkung & Remark \\
    \texttt{bemerkung} & Bemerkung & Remark \\
    \texttt{example} & Beispiel & Example \\
    \texttt{beispiel} & Beispiel & Example \\
    \texttt{proof} & Beweis & Proof \\
    \texttt{beweis} & Beweis & Proof \\
    \texttt{conjecture} & Vermutung & Conjecture \\
    \texttt{vermutung} & Vermutung & Conjecture \\
    \texttt{problem} & Problem & problem \\
\bottomrule

    
  \end{tabular}
  \caption{vordefinierte \texttt{ntheorem}-Umgebungen}
  \label{tab:ntheorem}
\end{table}

\subsection{booktabs}
\label{sec:intro:booktabs}
Mit diesem Paket lassen sich schönere Tabellen setzen, \cref{tab:intro:dateien}
zeigt ein Beispiel. Viele Hinweise zum Tabellensatz finden sich auch in der
ausführlichen Dokumentation zu diesem Paket.

\subsection{tabularx}
\label{sec:intro:tabularx}
Tabellen, welche die Breite bestimmer Spalten ändern, um eine vorgegebene Breite zu haben, können mit Hilfe von \texttt{tabularx} erzeugt werden.

\subsection{TikZ}
\label{sec:intro:tikz}
TikZ ist zwar kein Zeichenprogramm, man kann damit aber trotzdem verdammt gute
Abbildungen erzeugen. Die Dokumentation \cite{Tantau2007} ist sehr ausführlich,
im Internet findet man unter \url{http://www.texample.net} eine Menge
Beispiele, die zeigen, was mit dem Paket alles möglich ist. Damit das
Abbildungsverzeichnis nicht so leer bleibt, fügen wir hier in
\cref{fig:split-disjunction} mal ein TikZ-Bild
ein. Keine Angst, den Code müssen Sie nicht direkt verstehen. In der
TikZ-Anleitung gibt es auch ein paar schön erklärte, einfachere Beispiele.

\begin{figure}[htb]
\centering
\begin{subfigure}[b]{9cm}
\tikzset{%
  covering text/.style={shape=rectangle, fill=white, opaque, inner sep=2pt},%
  coordinate axis/.style={very thick, -stealth'},%
  faded coordinate axis/.style={gray, -stealth'},%
  coordinate grid/.style={help lines,xstep=1,ystep=1},%
  polytope/.style ={ultra thick},%
  integer polytope/.style={orange, ultra thick},%, 
  grid point/.style = {draw = none,fill=orange},%
  vertex point/.style = {draw = none,fill=black},%
  grid line/.style = {ultra thick,orange},%
  inequality hyperplane/.style={ultra thick, rot},%
  inequality halfspace/.style={draw=none, fill=rot, opacity=0.3},%
  valid hyperplane/.style={ultra thick, green},%
  valid halfspace/.style={draw=none, fill=green, opacity=0.3},%
  objective function/.style={ultra thick, gelb},%
  highlighted point/.style={draw=none,fill=blau},%
}
  \centering
    \begin{tikzpicture}[x=1.5cm, y=1.5cm]
    %\useasboundingbox (-0.5,-0.7) rectangle (4.5,3.9);%
    \draw[coordinate grid] (0,0) grid (4,3);
    \draw[coordinate axis] (0,0) -- (0,3.3) node[left] {$y$};%
    \draw[faded coordinate axis] (0,0) -- (4.3,0) node[below] {$x$};%
    \foreach \x in {1,2,3,4} {%
      \fill[] (\x,0) circle (0.07);
      \draw[ultra thick, dotted] (\x,-0.1) -- (\x,3.1);%
    }%
    %         %
    \coordinate (A) at (0.5,0.2);%
    \coordinate (B) at (3.5,0.5);%
    \coordinate (C) at (1.3,2.8);%
    \coordinate (A1) at (intersection of 1,0--1,1 and A--B);%
    \coordinate (A2) at (intersection of 2,0--2,1 and A--B);%
    \coordinate (A3) at (intersection of 3,0--3,1 and A--B);%
    \coordinate (B1) at (intersection of 1,0--1,1 and A--C);%
    \coordinate (B2) at (intersection of 2,0--2,1 and B--C);%
    \coordinate (B3) at (intersection of 3,0--3,1 and B--C);%
%  
    \draw[polytope] (A) -- (B) -- (C) -- cycle;%
    \draw[integer polytope, fill opacity=0.5] (A1) -- (A2) -- (A3) -- (B3)
    -- (B2) -- (B1) -- cycle;%
      %
    \draw[highlighted point] (C) circle (0.07) node[blau, above right]
    {$(x^*, y^*)$};%
    \draw[highlighted point] (C |- 0,0) circle (0.07) node[blau, below
    right] {$x^*$};%
    \draw[valid halfspace] (1,-0.2) rectangle (-1,3.2);%
    \draw[valid halfspace] (2,-0.2) rectangle (4,3.2);%
    \draw[valid hyperplane] (1,-0.2) -- (1,3.2);%
    \draw[valid hyperplane] (2,-0.2) -- (2,3.2);%
    \draw[inequality hyperplane] (-0.5,2) -- (4,3);%
    \fill[inequality halfspace] (C) -- (intersection of -0.5,2--4,3 and A--C)
    -- (intersection of -0.5,2--4,3 and B--C) -- cycle;%
    \draw[green] (1,3.2) node[covering text, above left]  {$d^Tx \le \delta$};%
    \draw[green] (2,3.2) node[covering text, above right]  {$d^Tx \ge \delta+1$};%
%
  \end{tikzpicture}
  \caption{Beispiel einer Split Disjunction}
  \label{fig:split-disjunction}
 \end{subfigure}
 \qquad
 \begin{subfigure}[b]{4cm}
   \centering
   \includegraphics[width=2cm]{TUMlMblau.png}
  \caption{Das TUM Logo}
  \label{fig:logo}
\end{subfigure}
\caption{Zwei Graphiken}
\label{fig:graphics}
\end{figure}

\subsection{subcaption}
\label{sec:intro:subcaption}
Um innerhalb einer Figure mehrere Subfigures zu erzeugen, kann man die folgende Umgebung aus dem Packet \texttt{subcaption} nutzen: \verb|\begin{subfigure} ... \end{subfigure}|. Hiermit kann man Subfigure und Subtables mit einer zu figures und tables identischen Syntax erzeugen.

\subsection{fixme}
\label{sec:intro:fixme}

Mit diesem Paket kann man Anmerkungen im Dokument erstellen, die noch nötige
Arbeiten dokumentieren.\fxnote{Hier könnte noch ein Beispiel folgen.} Ganz am
Ende des Dokuments gibt es dann eine \enquote{List of Corrections}, in der alle
Notizen aufgelistet sind. In diesem Absatz sind zur Demonstration zwei solche
FixMe-Hinweise eingebaut -- man sieht das einmal an den Anmerkungen am Rand,
andererseits aber auch ganz hinten in diesem Dokument in der erwähnten
\enquote{List of Corrections}. Auch hier kann man zahlreiche Einstellungen
vornehmen, die \fxnote*{ins Literaturerzeichnis aufnehmen}{Dokumentation} ist
empfehlenswert. Erwähnt sei eine Einstellung, die man ganz am Anfang der Datei
\texttt{thesis.tex} vornehmen sollte:
\begin{lstlisting}[language={[LaTeX]TeX}]
%FixMe-Status: final (keine FixMe-Anmerkungen) oder draft (Anmerkungen sichtbar)
\fxsetup{draft}
%\fxsetup{final}
\end{lstlisting}

Ersetzt man hier die \enquote{draft}-Zeile durch die \enquote{final}-Zeile, so
passieren zwei Dinge: Alle \verb|\fxfatal{}|-Kommandos werden zu \LaTeX-Fehlern,
der \TeX-Lauf bricht bei so einem Kommando also ab (nützlich, um wirklich böse
Fehler anzumerken, die man auf keinen Fall übersehen darf). Alle anderen
fixme-Kommandos (also \verb|\fxnote{}|, \verb|\fxwarning{}|, \verb|\fxerror{}|)
werden dagegen unsichtbar, die Markierungen im Text und auch die \enquote{List
  of Corrections} am Ende des Dokuments verschwinden. Genaueres zu den möglichen
Kommandos und zahlreichen Einstellmöglichkeiten bietet die Dokumentation
\cite{fixme}.

\subsection{hyperref}
\label{sec:intro:hyperref}

Mit dem \texttt{hyperref}-Paket werden einige PDF-Einstellungen vorgenommen
(vgl. \cref{sec:intro:grundeinstellungen}). Außerdem sorgt das Paket dafür, dass
alle Referenzen, Literaturverweise und das Inhaltsverzeichnis zu klickbaren
Links werden, mit denen man im Dokument hin- und herspringen kann. Standardmäßig
werden diese Links auch in schwarzer Farbe gesetzt, sie sind also nicht sofort
sichtbar. Alternativ kann man mit der Einstellung \texttt{colorlinks=true} im
\verb|\hypersetup{}|-Befehl zu Beginn des Dokuments aber auch dafür sorgen, dass
Links in einem dezenten dunkelblauen Farbton erscheinen. Für die
Bildschirm-Version ist das praktisch, für die Druckversion sollte man aber auf
schwarze Farbe zurückschalten, also die Option \texttt{colorlinks=false}
verwenden (schließlich kann man Links im Ausdruck ja nicht anklicken).
\begin{lstlisting}[language={[LaTeX]TeX}]
\hypersetup{
 pdfauthor={Wolfgang Ferdinand Riedl, Michael Ritter},
 pdftitle={Die tumthesis-Klasse},
 pdfsubject={Anleitung für Abschlussarbeiten},
 pdfkeywords={Masterarbeit, Bachelorarbeit},
 colorlinks=true, %farbige Links (für die PDF-Version)
% colorlinks=false, % keine farbigen Links (für die Druckversion)
}
\end{lstlisting}

\subsection{listings}
\label{sec:intro:listings}
Mit dem Paket \texttt{listings} kann man Quellcode-Listings schön formatiert
darstellen. In diesem Beispiel wird es benutzt, um \LaTeX-Quellcode
darzustellen. Mit den Standardeinstellungen in diesem Paket sorgt es automatisch
für Zeilennummerierung, Zeilenumbruch und einige anderen
Kleinigkeiten. Natürlich lässt sich vieles individuell anpassen, Details dazu
sind in der Dokumentation \cite{listings} zu finden. Eine kleine Warnung: Das
Paket ist so eingestellt, dass es mit Umlauten und \enquote{ß} in Quelltexten
zurechtkommt, andere Sonderzeichen können aber Probleme machen  (auch in
Kommentaren). Am besten, man vermeidet Sonderzeichen in Quelltexten komplett --
wenn es aber nicht anders geht, kann man sich in \texttt{tumthesis.cls}
anschauen, wie die Einstellungen ergänzt werden müssen, um auch andere
Sonderzeichen zu behandeln.

\subsection{algorithm2e}
\label{sec:intro:algorithm2e}
Mit \texttt{algorithm2e} hat man eine weitere Möglichkeit, Quelltext darzustellen. 
Im Gegensatz zu \texttt{listings} wird hier jedoch nicht Code für eine bestimmte
Programmiersprache formatiert, vielmehr wird formatierter Pseudocode mit
einer eigenen leicht verständlichen Syntax erzeugt.


%%% Local Variables: 
%%% mode: latex
%%% TeX-master: "thesis"
%%% End: 
